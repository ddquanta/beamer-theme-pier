% Set document class. 'c' and 't' refers to center and top vertical alignment, respectively.
\documentclass[11pt,aspectratio=169, c]{beamer} % 16:9 version
%\documentclass[11pt,aspectratio=43, t]{beamer} % 4:3 version
% \documentclass[11pt,t]{beamer} % default, 4:3 version



% packages
\usepackage[english]{babel}
%\usepackage{geometry}
\usepackage[utf8]{inputenc}
\usepackage{mathtools}
\usepackage{graphicx}
\usepackage{amssymb}
\usepackage{amsmath}
\DeclareMathOperator*{\argmax}{arg\,max}
\usepackage{bbold}
\usepackage{float}
\usepackage{subcaption}
%\usepackage{hyperref}
\usepackage[backend=biber,style=apa,citestyle=authoryear-comp]{biblatex}
%\usepackage{natbib}
\DeclareMathOperator{\E}{\mathbb{E}}

%\addbibresource{literature.bib}   % bibtex file
\usepackage[absolute,overlay]{textpos}
\usepackage{color}
\usepackage{xcolor}
\usepackage{multirow}
\usepackage{pgffor}

%%%%%%%%%%%%%%%%%%%%%%%%%%%%%%%%%%%%%%%%%%%%%%%%%%%%%%%%%%%%

\BeforeBeginEnvironment{frame}{%
	\setbeamertemplate{background}{
		\begin{tikzpicture}
			\useasboundingbox (0,0) rectangle(\paperwidth, \paperheight);
			\node[inner sep=0,outer sep=0, anchor=south east] (fold) at (\paperwidth, 0) {\includegraphics[width=1.2cm]{resources/fold.pdf}};
		\end{tikzpicture}
	}
	\setbeamertemplate{frametitle}[PIER]%
	\setbeamertemplate{footline}[PIER]%
	%\setbeamertemplate{footline}{\hfill\usebeamercolor{white}\insertframenumber{}}
	\normalsize%
	\normalcolor%
}

\makeatletter

\define@key{beamerframe}{paper}[true]{%
	\setbeamertemplate{background}{
		\begin{tikzpicture}
			\usebeamercolor{palette primary}
			\useasboundingbox (0,0) rectangle (\paperwidth, \paperheight);
			\fill[fill=fg] (0, 0) rectangle (\paperwidth, \paperheight);
			\fill[white] (0.5, 0.5) rectangle (\paperwidth - 0.5cm, \paperheight - 0.5cm);
		\end{tikzpicture}
	}
	\setbeamertemplate{frametitle}{
		\begin{beamercolorbox}[wd=\paperwidth, ht=1.5cm, center]{frametitle}
			\insertframetitle
		\end{beamercolorbox}
	}
	\setbeamertemplate{footline}{}
}

\makeatother

\setbeamertemplate{page number in head/foot}[framenumber]

%%%%%%%%%%%%%%%%%%%%%%%%%%%%%%%%%%%%%%%%%%%%%%%%%%%%%%%%%%%%

\usetheme{PIER}
%\usetheme{Singapore}

%%%%%%%%%%%%%%%%%%%%%%%%%%%%%%%%%%%%%%%%%%%%%%%%%%%%%%%%%%%%

% Title page.

\title[V2]{Template Example ver. 2}
\author[Mini author list]{Don Tawanpitak}
\date{\today}

\usetheme{PIER}

%%%%%%%%%%%%%%%%%%%%%%%%%%%%%%%%%%%%%%%%%%%%%%%%%%%%%%%%%%%%

\begin{document}
	\begin{frame}[paper]
		
		\titlepage
		
		\begin{center}
			\includegraphics[height=1.2cm]{resources/pier-logo.pdf}
		\end{center}
		
	\end{frame}
	
%%%%%%%%%%%%%%%%%%%%%%%%%%%%%%%%%%%%%%%%%%%%%%%%%%%%%%%%%%%%
	
% Comment this part if a section page is not needed.
	
\AtBeginSection[]{
	\begin{frame}
		
		\frametitle{\mbox{}}
		\vfill
		\centering
		\begin{beamercolorbox}[
			wd  = \paperwidth,
			sep = 1.2ex, center, shadow = true, rounded = true]{title}
			\usebeamerfont{title}\huge\insertsectionhead
		\end{beamercolorbox}
		\vfill
		
	\end{frame}
}
	
%%%%%%%%%%%%%%%%%%%%%%%%%%%%%%%%%%%%%%%%%%%%%%%%%%%%%%%%%%%%
	
	% This part set the rules for spacing between paragraphs.
	\setlength{\parskip}{2em}
	
	
	
	% This part set the rules for spacing between items.
	\BeforeBeginEnvironment{itemize}{
		\par
		\vspace*{-\parskip}
		\vspace{0.2em}
	}
	
	\AtBeginEnvironment{itemize}{
		\setlength{\itemsep}{0.2em}
	}
	
	
	
	% This part set the rules for spacing between items.
	\BeforeBeginEnvironment{enumerate}{
		\par
		\vspace*{-\parskip}
		\vspace{0.2em}
	}
	
	\AtBeginEnvironment{enumerate}{
		\setlength{\itemsep}{0.2em}
	}
	
%%%%%%%%%%%%%%%%%%%%%%%%%%%%%%%%%%%%%%%%%%%%%%%%%%%%%%%%%%%%

	\section{Introduction}

	\begin{frame}{Motivation}
		
		This .tex example is based on P'Art's original template but with some modifications.
		
		First, the mini author list (bottom left) can now be as long as needed. The footline will auto-adjust its width accordingly. The same applies for the section part of the footline.
		
		Second, the section page, i.e., page 2, will now auto adjust text position to the center. If that page is not needed, then comment that part in the .tex file.
		
		Third, a new part is added to the .tex file for setting space between, e.g., paragraphs and items. This can be adjusted, or entirely commented, per the author's preferences.
		
	\end{frame}



\end{document}
